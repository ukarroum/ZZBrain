\section{Quelques Applications}

Nous présenterons dans cette section des exemples concrets illustrant l'utilisaton de ZZBrain, ce ne sont que des fonctions logiques, ils n'ont donc aucun intérét pratique puisqu'ils peuvent être codé plus aisement de façon directe.

Nous tenons à préciser un point, il peut sembler évident qu'une dataset contenant tous les cas possibles (2 pour une fonction unaire, 4 pour une fonction binaire) sera largement suffisante, mais apparement (résultats expérimentaux) il s'avére que le réseau est beaucoup plus performent si on lui fournit une dataset plus large (par exemple de taille 1000), même si celle-ci ne contient en réalité que 2 entrés répétés plusieurs fois.
Pour être tout à fait honnete, nous ne savons toujours pas pourquoi c'est le cas.

\subsection{Réseaux sans couches intérmédiaires}

\subsubsection{Non logique}

\begin{minted}[linenos=true, tabsize=4, fontfamily=courier, fontsize=\small, xleftmargin=5pt, xrightmargin=5pt]{cpp}
#include <iostream>
#include "lib/ZZNetwork.h"

using namespace std;

int main() {

	int sizes[] = {1, 1};
	int nbLayers = 2;

	double **X = new double*[1000];
	double **Y = new double*[1000];


	for(int i = 0; i < 1000; i++)
	{
		X[i] = new double[1];
		Y[i] = new double[1];
	}


	for(int i = 0; i < 500; i++)
	{
		X[i][0] = 0.0;
		Y[i][0] = 1.0;
	}
	for(int i = 500; i < 1000; i++)
	{
		X[i][0] = 1.0;
		Y[i][0] = 0.0;
	}

	ZZNetwork notNet(sizes, nbLayers, 1000, X, Y);
	notNet.train();

	cout << "not(0) = " << notNet.predict(X[0])[0] << endl;
	cout << "not(1) = " << notNet.predict(X[999])[0] << endl;

    return 0;

}
\end{minted}

\begin{figure}[h]
\centering
    \begin{tikzpicture}[shorten >=1pt,->,draw=black!50, node distance=\layersep]
        \tikzstyle{every pin edge}=[<-,shorten <=1pt]
        \tikzstyle{neuron}=[circle,fill=black!25,minimum size=17pt,inner sep=0pt]
        \tikzstyle{input neuron}=[neuron, fill=green!50];
        \tikzstyle{output neuron}=[neuron, fill=red!50];
        \tikzstyle{hidden neuron}=[neuron, fill=blue!50];
        \tikzstyle{annot} = [text width=4em, text centered]

        % Draw the input layer nodes
        %\foreach \name / \y in {1,...,2}
        % This is the same as writing \foreach \name / \y in {1/1,2/2,3/3,4/4}
            \node[input neuron, pin=left: $+1$] (I-1) at (0,-1) {};
            \node[input neuron, pin=left: $x$] (I-2) at (0,-2) {};

        % Draw the output layer node
        \node[output neuron,pin={[pin edge={->}]right:$h_\Theta(x)$}, right of=I-2] (O) {};

        \path (I-1) edge node[above, scale=0.7]{$4.09538$}(O);
        \path (I-2) edge node[below, scale=0.7]{$-8.21579$}(O);


    \end{tikzpicture}
    \caption{Réseau Non logique généré par ZZBrain}

\end{figure}

\subsubsection{And logique}

\begin{minted}[linenos=true, tabsize=4, fontfamily=courier, fontsize=\small, xleftmargin=5pt, xrightmargin=5pt]{cpp}
#include <iostream>
#include "lib/ZZNetwork.h"

using namespace std;

int main() {

	int sizes[] = {2, 1};
	int nbLayers = 2;

	double **X = new double*[1000];
	double **Y = new double*[1000];


	for(int i = 0; i < 1000; i++)
	{
		X[i] = new double[2];
		Y[i] = new double[2];
	}


	for(int i = 0; i < 250; i++)
	{
		X[i][0] = 0.0;
		X[i][1] = 0.0;
		Y[i][0] = 0.0;
	}
	for(int i = 250; i < 500; i++)
	{
		X[i][0] = 1.0;
		X[i][1] = 0.0;
		Y[i][0] = 0.0;
	}
	for(int i = 500; i < 750; i++)
	{
		X[i][0] = 0.0;
		X[i][1] = 1.0;
		Y[i][0] = 0.0;
	}
	for(int i = 750; i < 1000; i++)
	{
		X[i][0] = 1.0;
		X[i][1] = 1.0;
		Y[i][0] = 1.0;
	}



	ZZNetwork notNet(sizes, nbLayers, 1000, X, Y);
	notNet.train();

	cout << "and(0, 0) = " << notNet.predict(X[0])[0] << endl;
	cout << "and(1, 0) = " << notNet.predict(X[300])[0] << endl;
	cout << "and(0, 1) = " << notNet.predict(X[600])[0] << endl;
	cout << "and(1, 1) = " << notNet.predict(X[900])[0] << endl;

    return 0;

}
\end{minted}

\begin{figure}[h]
\centering
    \begin{tikzpicture}[shorten >=1pt,->,draw=black!50, node distance=\layersep]
        \tikzstyle{every pin edge}=[<-,shorten <=1pt]
        \tikzstyle{neuron}=[circle,fill=black!25,minimum size=17pt,inner sep=0pt]
        \tikzstyle{input neuron}=[neuron, fill=green!50];
        \tikzstyle{output neuron}=[neuron, fill=red!50];
        \tikzstyle{hidden neuron}=[neuron, fill=blue!50];
        \tikzstyle{annot} = [text width=4em, text centered]

        % Draw the input layer nodes
        %\foreach \name / \y in {1,...,2}
        % This is the same as writing \foreach \name / \y in {1/1,2/2,3/3,4/4}
            \node[input neuron, pin=left: $+1$] (I-1) at (0,-1) {};
            \node[input neuron, pin=left: $x_1$] (I-2) at (0,-2) {};
            \node[input neuron, pin=left: $x_2$] (I-3) at (0,-3) {};

        % Draw the output layer node
        \node[output neuron,pin={[pin edge={->}]right:$h_\Theta(x)$}, right of=I-2] (O) {};

        \path (I-1) edge node[above, scale=0.7]{$-66.052$}(O);
        \path (I-2) edge node[above, scale=0.7]{$40.2334$}(O);
        \path (I-3) edge node[below, scale=0.7]{$50.7644$}(O);


    \end{tikzpicture}
    \caption{Réseau And logique généré par ZZBrain}

\end{figure}

\subsection{Réseaux avec couches intérmédiaires}

\subsubsection{XNOR logique}
\begin{minted}[linenos=true, tabsize=4, fontfamily=courier, fontsize=\small, xleftmargin=5pt, xrightmargin=5pt]{cpp}
#include <iostream>
#include "lib/ZZNetwork.h"

using namespace std;

int main() {

	int sizes[] = {2, 4, 1};
	int nbLayers = 3;

	double **X = new double*[1000];
	double **Y = new double*[1000];


	for(int i = 0; i < 1000; i++)
	{
		X[i] = new double[2];
		Y[i] = new double[2];
	}


	for(int i = 0; i < 250; i++)
	{
		X[i][0] = 0.0;
		X[i][1] = 0.0;
		Y[i][0] = 1.0;
	}
	for(int i = 250; i < 500; i++)
	{
		X[i][0] = 1.0;
		X[i][1] = 0.0;
		Y[i][0] = 0.0;
	}
	for(int i = 500; i < 750; i++)
	{
		X[i][0] = 0.0;
		X[i][1] = 1.0;
		Y[i][0] = 0.0;
	}
	for(int i = 750; i < 1000; i++)
	{
		X[i][0] = 1.0;
		X[i][1] = 1.0;
		Y[i][0] = 1.0;
	}



	ZZNetwork notNet(sizes, nbLayers, 1000, X, Y);
	notNet.train();

	cout << "xnor(0, 0) = " << notNet.predict(X[0])[0] << endl;
	cout << "xnor(1, 0) = " << notNet.predict(X[300])[0] << endl;
	cout << "xnor(0, 1) = " << notNet.predict(X[600])[0] << endl;
	cout << "xnor(1, 1) = " << notNet.predict(X[900])[0] << endl;

    return 0;

}
\end{minted}

\begin{figure}[h]
\centering
    \begin{tikzpicture}[shorten >=1pt,->,draw=black!50, node distance=\layersep]
        \tikzstyle{every pin edge}=[<-,shorten <=1pt]
        \tikzstyle{neuron}=[circle,fill=black!25,minimum size=17pt,inner sep=0pt]
        \tikzstyle{input neuron}=[neuron, fill=green!50];
        \tikzstyle{output neuron}=[neuron, fill=red!50];
        \tikzstyle{hidden neuron}=[neuron, fill=blue!50];
        \tikzstyle{annot} = [text width=4em, text centered]

        % Draw the input layer nodes
        \node[input neuron, pin=left: $+1$] (I-1) at (0,-1) {};
        \node[input neuron, pin=left: $x_1$] (I-2) at (0,-2) {};
        \node[input neuron, pin=left: $x_2$] (I-3) at (0,-3) {};


        % Draw the hidden layer nodes
        \foreach \name / \y in {1,...,4}
            \path[yshift=0.5cm]
                node[hidden neuron] (H-\name) at (\layersep,-\y cm) {$a_\y^{(1)}$};

        % Draw the output layer node
        \node[output neuron,pin={[pin edge={->}]right:$h_\Theta(x)$}, right of=H-3] (O) {};

        % Connect every node in the input layer with every node in the
        % hidden layer.
        \foreach \source in {1,...,3}
            \foreach \dest in {1,...,4}
                \path (I-\source) edge (H-\dest);

        % Connect every node in the hidden layer with the output layer
        \foreach \source in {1,...,4}
            \path (H-\source) edge (O);

    \end{tikzpicture}
    \caption{Réseau XNOR généré par ZZBrain}

\end{figure}

Les poids générés sont données dans la matrice suivante :

\begin{gather*}

\Theta^{(0)} = \begin{pmatrix}
  -1.73473  & -1.75034 & -5.93952 & -5.48223\\
  33.4988 & -46.5189 & -1.83867 & -19.2062  \\
  16.4799 & 29.3001 & -0.18966 & -37.6205  \\
 \end{pmatrix}

 \Theta^{(1)} = \begin{pmatrix}
 -43.7497 & -85.383 & 3.75306 \\
  \end{pmatrix}
\end{gather*}
